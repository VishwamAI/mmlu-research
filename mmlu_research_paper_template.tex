\documentclass[journal]{IEEEtran}
\usepackage{amsmath,amsfonts}
\usepackage{algorithmic}
\usepackage{array}
\usepackage[caption=false,font=normalsize,labelfont=sf,textfont=sf]{subfig}
\usepackage{textcomp}
\usepackage{stfloats}
\usepackage{url}
\usepackage{verbatim}
\usepackage{graphicx}
\usepackage{cite}
\hyphenation{op-tical net-works semi-conduc-tor IEEE-Xplore}

\begin{document}

\title{How to Reach MMLU 100\% Accuracy}
\author{Devin AI and Kasinadhsarma
\thanks{Manuscript created June, 2024; This work was developed by Devin AI and Kasinadhsarma. The opinions expressed here are entirely those of the authors. No warranty is expressed or implied. User assumes all risk.}}

\maketitle

\begin{abstract}
This research paper explores the methodologies and strategies required to achieve 100\% accuracy on the Massive Multitask Language Understanding (MMLU) benchmark. By reviewing state-of-the-art models, analyzing their architectures, and discussing key challenges and techniques, we aim to provide a comprehensive roadmap for reaching this milestone.
\end{abstract}

\begin{IEEEkeywords}
MMLU, Language Models, Multimodal Training, Multilingual Training, Efficient Attention Mechanisms, Reinforcement Learning from Human Feedback (RLHF), Chain-of-Thought Prompting
\end{IEEEkeywords}

\section{Introduction}
\IEEEPARstart{T}{he} Massive Multitask Language Understanding (MMLU) benchmark is a comprehensive evaluation of a model's ability to perform a wide range of language tasks. Achieving 100\% accuracy on this benchmark is a significant milestone that demonstrates a model's capability to understand and process language at a human-expert level. This paper aims to explore the methodologies and strategies required to reach this goal. We will review the current state-of-the-art models, analyze their architectures, and discuss the key challenges and techniques involved in achieving 100\% accuracy on the MMLU benchmark.

\section{Related Work}
The MMLU benchmark has seen significant advancements in recent years, with several models achieving remarkable performance. Notable among these are the Gemini Ultra ~1760B, GPT-4o, and Claude 3 Opus models. The Gemini Ultra ~1760B model, in particular, has set a new standard by achieving an average performance of 90\% on the MMLU benchmark. This section reviews the existing models, their performance metrics, and key papers and benchmarks that have contributed to the progress in this field.

\subsection{MMLU-Pro Benchmark}
MMLU-Pro is an enhanced dataset designed to extend the MMLU benchmark by integrating more challenging, reasoning-focused questions and expanding the choice set from four to ten options. It aims to address the performance plateau of large-scale language models on existing benchmarks and reduce the sensitivity of model scores to prompt variations. MMLU-Pro spans 14 diverse domains, including over 12,000 questions, and is more discriminative in distinguishing nuances between models. The leading model, GPT-4o, achieves an accuracy of 72.6\% on MMLU-Pro, indicating substantial room for improvement. MMLU-Pro necessitates chain-of-thought reasoning for better performance, contrasting with MMLU where this approach may be detrimental. Error analysis of GPT-4o shows that the majority of errors are due to reasoning flaws, lack of domain knowledge, and computational mistakes, highlighting areas for future research and model development.

\section{Methodology}
This section describes the models analyzed, their architectures, training techniques, and the datasets used. We also integrate images and model designs to provide a comprehensive understanding of the methodologies employed.

\subsection{Key Strategies and Techniques}
This subsection compiles the key strategies and techniques used by the top-performing models analyzed in this research.

\subsubsection{Gemini Ultra ~1760B}
\begin{itemize}
    \item \textbf{Multimodal and Multilingual Training}: Training on a diverse dataset that includes text, image, audio, and video data to enhance cross-modal understanding.
    \item \textbf{Efficient Attention Mechanisms}: Utilizing multi-query attention and other efficient attention mechanisms to support long context lengths.
    \item \textbf{Advanced Training Infrastructure}: Leveraging TPUv5e and TPUv4 accelerators across multiple datacenters for stable large-scale training and optimized inference.
\end{itemize}
\begin{figure}[htbp]
\centerline{\includegraphics[width=0.5\textwidth]{gemini_ultra_architecture.png}}
\caption{Gemini Ultra Architecture}
\label{fig:gemini_ultra}
\end{figure}

\subsubsection{GPT-4o}
\begin{itemize}
    \item \textbf{Multimodal Capabilities}: Processing both text and image inputs to produce text outputs, expanding the model's application scope.
    \item \textbf{Predictable Scaling}: Developing infrastructure and optimization methods for predictable behavior across multiple scales.
    \item \textbf{Reinforcement Learning from Human Feedback (RLHF)}: Fine-tuning the model using RLHF to enhance factuality and adherence to desired behaviors.
\end{itemize}

\subsubsection{Claude 3 Opus}
\begin{itemize}
    \item \textbf{Multimodal Architecture}: Integrating vision capabilities to process and analyze image data, enhancing reasoning, math, and coding capabilities.
    \item \textbf{Constitutional AI}: Aligning the model with human values through a set of ethical and behavioral principles.
    \item \textbf{Comprehensive Safety Measures}: Implementing an Acceptable Use Policy (AUP), real-time classifiers, and regular risk assessments to ensure safety and ethical AI development.
\end{itemize}
\begin{figure}[htbp]
\centerline{\includegraphics[width=0.5\textwidth]{claude_3_opus_architecture.png}}
\caption{Claude 3 Opus Architecture}
\label{fig:claude_3_opus}
\end{figure}

\subsubsection{Leeroo Orchestrator}
\begin{itemize}
    \item \textbf{LLM-Based Orchestrator}: Selecting the right underlying LLM experts for optimal task execution.
    \item \textbf{Synthetic Data Generation via Self-Play}: Creating training data through a loop of query generation, orchestration, and evaluation.
    \item \textbf{Cost-Effective Performance}: Achieving higher accuracy at a lower cost compared to existing models by optimizing the synergy between multiple LLMs.
\end{itemize}
\begin{figure}[htbp]
\centerline{\includegraphics[width=0.5\textwidth]{leeroo_architecture.png}}
\caption{Leeroo Orchestrator Architecture}
\label{fig:leeroo}
\end{figure}

\section{Results}
\begin{figure}[htbp]
\centerline{\includegraphics[width=0.5\textwidth]{performance_metrics_graphs.png}}
\caption{Performance Metrics Graphs}
\label{fig:performance_metrics}
\end{figure}

\section{Discussion}
Achieving 100\% accuracy on the MMLU benchmark presents several key challenges
\end{document}
